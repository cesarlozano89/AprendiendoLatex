\documentclass[10pt,a4paper]{article}

\usepackage[utf8]{inputenc}
\usepackage[spanish]{babel}
\usepackage{graphicx}

\author{Vikman}
\title{Ejercicio}

\begin{document}

  \maketitle
  \tableofcontents
  \listoffigures

  \pagebreak
  \section{Ejemplo de lista}

  \begin{itemize}
    \item Primer elemento.
    \item Segundo elemento.
    \item Tercer elemento.
  \end{itemize}
  
  \ldots Y así podríamos seguir muchas páginas.

  \section{Gráfico}

  \setlength{\unitlength}{0.8cm}
  \begin{figure}[!ht]
    \centering
    \begin{tabular}{|p{\textwidth}|}
      \hline
      \begin{picture}(6,4)(-3,-2)
        \put(-2.5,0){\vector(1,0){5}}
        \put(2.7,-0.1){$\chi$}
        \put(0,-1.5){\vector(0,1){3}}
        \multiput(-2.5,1)(0.4,0){13}{\line(1,0){0.2}}
        \multiput(-2.5,-1)(0.4,0){13}{\line(1,0){0.2}}
        \put(0.2,1.4){$\beta=v/c=\tanh\chi$}
        \qbezier(0,0)(0.8853,0.8853)(2,0.9640)
        \qbezier(0,0)(-0.8853,-0.8853)(-2,-0.9640)
        \put(-3,-2){\circle*{0.2}}
      \end{picture} \\
      \hline
    \end{tabular}
    
    \caption[Relatividad especial de Einstein]{\label{fig:einstein} Relatividad especial.}
  \end{figure}

  \section{Imagen}
  
  \begin{figure}[!ht]
    \includegraphics[width=\textwidth]{imagen}
    \caption[Ejemplo de imagen]{\label{fig:im} Una imagen cualquiera.}
  \end{figure}

\end{document}