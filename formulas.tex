\documentclass[10pt,a4paper]{article}
\usepackage[utf8]{inputenc}
\usepackage[spanish]{babel}
\usepackage{amsmath}
\author{Vikman}
\title{Mi documento}
\begin{document}

\section{Ecuaciones de Maxwell}

\begin{equation}
\Phi=\oint_{S}\overrightarrow{E} \cdot d\overrightarrow{S}=\frac{q}{\varepsilon_0} \quad \text{(Ley de Gauss)}
\end{equation}

\begin{equation}
\oint_S\overrightarrow{B} \cdot d\overrightarrow{S}=0 \quad\text{(Ley de Gauss para el campo magnético)}
\end{equation}

\begin{equation}
\oint_C \overrightarrow{E} \cdot d\overrightarrow{l} = - \frac{d}{dt} \int_S \overrightarrow{B} \cdot d\overrightarrow{S} \quad \text{(Ley de Faraday)}
\end{equation}

\begin{equation}
\oint_C \overrightarrow{B} \cdot d\overrightarrow{l}=\mu_0 \int_S \overrightarrow{j} \cdot d\overrightarrow{S} + \mu_0 \epsilon_0 \frac{d}{dt} \int_S \overrightarrow{E} \cdot d\overrightarrow{S} \quad \text{(Ley de Ampère)}
\end{equation}

\begin{equation}
\begin{pmatrix}
0 & E_x/c & E_y/c & E_x/c \\
-E_x/c & 0 & -B_z & B_y \\
-E_y/c & B_z & 0 & -B_x \\
-E_z/c & -B_y & B_x & 0
\end{pmatrix}
\end{equation}

\end{document}